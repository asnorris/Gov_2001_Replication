% !TeX program = pdfLaTeX
\documentclass[12pt]{article}
\usepackage{amsmath}
\usepackage{graphicx,psfrag,epsf}
\usepackage{enumerate}
\usepackage{natbib}
\usepackage{textcomp}
\usepackage[hyphens]{url} % not crucial - just used below for the URL
\usepackage{hyperref}
\providecommand{\tightlist}{%
  \setlength{\itemsep}{0pt}\setlength{\parskip}{0pt}}

%\pdfminorversion=4
% NOTE: To produce blinded version, replace "0" with "1" below.
\newcommand{\blind}{0}

% DON'T change margins - should be 1 inch all around.
\addtolength{\oddsidemargin}{-.5in}%
\addtolength{\evensidemargin}{-.5in}%
\addtolength{\textwidth}{1in}%
\addtolength{\textheight}{1.3in}%
\addtolength{\topmargin}{-.8in}%

%% load any required packages here



% Pandoc citation processing

\usepackage{booktabs}
\usepackage{longtable}
\usepackage{array}
\usepackage{multirow}
\usepackage{wrapfig}
\usepackage{float}
\usepackage{colortbl}
\usepackage{pdflscape}
\usepackage{tabu}
\usepackage{threeparttable}
\usepackage{threeparttablex}
\usepackage[normalem]{ulem}
\usepackage{makecell}
\usepackage{xcolor}
\usepackage{dcolumn}

\begin{document}


\def\spacingset#1{\renewcommand{\baselinestretch}%
{#1}\small\normalsize} \spacingset{1}


%%%%%%%%%%%%%%%%%%%%%%%%%%%%%%%%%%%%%%%%%%%%%%%%%%%%%%%%%%%%%%%%%%%%%%%%%%%%%%

\if0\blind
{
  \title{\bf Rainfall, Drought Declaration, and Reelection: A
replication of \citet{cooperman2021natural}}

  \author{
        Scott Mahon \thanks{The authors gratefully acknowledge the help
of TFs Soubhik Barari and Chris Kenny, Professor Gary King, and the
reviewers in the writing of this paper.} \\
    Harvard College\\
     and \\     Alexandra Norris \thanks{Replication data can be
accessed
\href{https://dataverse.harvard.edu/dataset.xhtml?persistentId=doi:10.7910/DVN/P6WM4E}{HERE}} \\
    Harvard College\\
      }
  \maketitle
} \fi

\if1\blind
{
  \bigskip
  \bigskip
  \bigskip
  \begin{center}
    {\LARGE\bf Rainfall, Drought Declaration, and Reelection: A
replication of \citet{cooperman2021natural}}
  \end{center}
  \medskip
} \fi

\bigskip
\begin{abstract}
In a recent study, \citet{cooperman2021natural} finds that among mayors
in northeastern Brazil, the declaration of a drought is associated with
an increased likelihood of being reelected and that periods of below
average rainfall are associated with an increased likelihood of
incumbents running for and winning reelection. In our analysis of the
paper's findings, we were able to replicate all results. We extended the
analysis through graphically depicting the relationship between
term-limits and drought declaration, exploring whether changing the time
period of analysis impacts the relationship between rainfall and drought
declaration, and finally through observing the effects of replacing a
binary variable for rainfall with a continuous scale. While we found
minor differences in results depending on how variables are interpreted
and whether binary or continuous variables are used for rainfall,
nothing we found refutes the paper's findings.
\end{abstract}

\noindent%
{\it Keywords:} Disaster, Drought, Reelection
\vfill

\newpage
\spacingset{1.45} % DON'T change the spacing!

\hypertarget{introduction}{%
\section{Introduction}\label{introduction}}

Understanding how politicians respond to upcoming elections can offer
profound insight into the political decision-making process and
incentive structure. For one, it shows how politicians interact with
their constituents in an election year compared to non-election years.
It also highlights how elected officials think about budgets, government
funding, and natural disasters, and what might incentivize them to make
greater investments in public services and resources. Through comparing
politicians' decisions in election years to non-election years,
\citet{cooperman2021natural} concurs with the existing body of
literature
\citep{alesina1989politics, finan2005reelection, martinez2009theory}
that politicians respond to reelection incentives and will distribute
public resources in the hopes of serving longer in office.

It's important to understand the political landscape of Brazil to fully
interpret the results of the paper. Mayors in Brazil are eligible for
two consecutive terms of four years each decided through a majority
vote. It is important to note that the Brazilian party system is very
fragmented and that social context and family ties often matter more
than party affiliation. There are many different parties, and some
mayors even switch parties between elections \citep{novaes2018disloyal}
--- this creates a political landscape where politicians care less about
party re-election after serving two terms\footnote{This is different
  than other country contexts like the U.S., where party lines are often
  much more solid} meaning that actions taken by term-limited
politicians likely are not motivated by the desire for a co-partisan to
be elected \citep{franzese2002electoral, clark1998international}. For
further context, northeast Brazil is one of the poorest regions in the
country with significant inequality both in state capacity and poverty
\citep{tendler1997good, finan2009decentralized, ottonelli2014pobreza}.

Disaster relief is funded by the state government and is either all or
nothing. To begin the drought related aid process, mayors must declare a
state of emergency and apply for approval from the governor. This
requires reporting of information on the extent of the drought, its
impact on residents, and how the relief money will be spent.
\citet{cooperman2021natural} did extensive research and field interviews
to fully understand the complexities of the Brazilian political scheme
and process for applying for drought funding.

\citet{cooperman2021natural} explores the relationship between municipal
drought declarations and mayoral reelections in northeast Brazil, the
country's predominantly agricultural region. Droughts are not uncommon
in Brazil and often disproportionately affect poor subsistence farmers.
When an emergency or drought is declared and approved by the national
government, resources are deployed to the affected communities.
\citet{cooperman2021natural} argues that these aid resources can be used
to garner support for incumbent politicians as a form of clientelism or
distributive politics, incentivizing politicians to declare droughts in
election years. Using both rainfall and electoral data,
\citet{cooperman2021natural} finds evidence that mayors in election
years are more likely to declare droughts, even in times of above
average rainfall, and that incumbent mayors are more likely to win
reelection when they declare droughts. Interestingly,
\citet{cooperman2021natural} also finds that second-term mayors, mayors
that are term limited and cannot run for reelection, declare droughts at
higher rates, potentially contradicting the argument that this may be an
attempt by politicians to buy votes. Importantly,
\citet{cooperman2021natural} contextualizes this analysis with
interviews with both Brazilian farmers and political figures, providing
a more detailed picture of what is happening beyond the data.

\section{Understanding the Impact of Term Limits}
\label{sec:term_limit}

In our replication of \citet{cooperman2021natural}, we were able to
successfully reproduce the paper's results and tables\footnote{To
  observe the original tables and figures, please visit the Appendix}.
As mentioned earlier, we were interested in looking more closely at
term-limited politicians and their drought declaring behaviors,
specifically why term-limited mayors declare droughts more frequently
than not term-limited mayors. \citet{cooperman2021natural} posits that
mayors become more experienced over time and that this experience makes
them better able to go through the relatively lengthy and burdensome
administrative processes that declaring drought emergencies require.

\begin{figure}[H]
  \caption{Drought Declaration and Term Limits}
  \includegraphics{Exhibits/figure1.pdf}
\end{figure}

To better observe the variation between mayors who are term limited and
those who are not, we create a graphical representation of Table 2 from
\citet{cooperman2021natural}. Instead of using the categorical variables
of ``below average'', ``above average'' and ``very high'' to represent
rainfall, we use the continuous SPI scale provided in the replication
data which represents the rainfall between January and June of that year
as it relates to the average rainfall from 1981-2012 during that
six-month period.\footnote{SPI ranges from -3 to 3 with numbers between
  0 and -0.79 indicating an abnormally dry period; numbers between -0.8
  and -1.29 indicating moderate drought; -1.3 to 1.59 indicating severe
  drought; -1.6 to -1.99 indicating extreme drought; and less than -2
  indicating exceptional drought. The same gradations can be viewed in
  the positive direction but for above average rainfall
  \citep{de2016drought}.}

Figure 1 shows the relationship between rainfall and drought
declarations for both term limited and non-term limited mayors. We chose
to display this relationship using smoothed conditional means due to the
binary nature of the dependent variable. Figure 1 shows that for
rainfall amounts that fall beneath 1, meaning they are above average,
incumbent mayors are significantly more likely to declare droughts than
their non-term limited counterparts. Observing the graph, there seems to
be a switch that happens after 1 but given the large confidence
intervals at that point, this switch is not statistically significant
and thus nothing should be inferred from it. Although perhaps counter
intuitive, as mentioned above, \citet{cooperman2021natural} credits this
relationship to the burdensome paperwork and legislative processes
associated with asking for drought relief, making experienced
politicians more likely to take on this burden.

Interestingly, in our replication of Table 2 of the original paper, we
find a nebulously labeled variable that could interfere with the
interpretation of results from both our paper and
\citet{cooperman2021natural}. The variable \citet{cooperman2021natural}
used to represent second term or term limited politicians is labelled
``lame duck'' in the provided replication dataset. Being term limited
and being a lame duck are not the same --- a lame duck is classified as
an official in the final period of office, after the election of a
successor and is not necessarily a term limited or second term
politician \citep{jenkins2008partisanship}. Although it may seem subtle,
this change in wording could have profound implications on the actual
meaning of the paper's results; the replication code and data available
did not include a code book for the dataset, so we are unsure if the
lame duck variable is indeed about lame ducks or simply term limited
politicians. If the former is true then the analysis in our paper and
\citet{cooperman2021natural} needs revising.

To further explore the true definition of the lame duck variable, we
found the original dataset issued by the Brazilian \emph{Tribunal
Superior Eleitoral}. Despite finding this, the dataset was in
Portuguese, so we were unable to competently interpret it. However, we
were able to find a code book from \citet{hollyer2021parties}, which
also used data from \emph{Tribunal Superior Eleitoral}, the same
political source as \citet{cooperman2021natural}. This codebook includes
a lame duck variable but defines it as indicating incumbency. This
finding further adds to our confusion about the true meaning of this
variable. Both lame duck definitions from \citet{cooperman2021natural}
and \citet{hollyer2021parties} are quite different from the true meaning
of the term ``lame duck''. Given that \citet{hollyer2021parties} and
\citet{cooperman2021natural} likely sourced their data from the same
website, we have reason to believe that there may have been some sort of
loss in translation or mis-definition causing this confusion. As a
result, the findings displayed in Table 2 of the original paper may not
actually represent what \citet{cooperman2021natural} argues they do.

\section{Rainfall Period and Drought Declaration}
\label{sec:rain_period}

\citet{cooperman2021natural} explores the relationship between rainfall,
whether it is an election year, and the declaration of droughts amongst
mayors in municipalities. As we replicated the paper's code, we
discovered that \citet{cooperman2021natural} used rainfall over the
six-month period between January and June as the reference for whether
or not there was a drought and how rainfall compared to the average
rainfall during that time. \citet{cooperman2021natural} justifies this
choice explaining that this six-month period is the typical ``rainy
season'' when farmers typically plant crops, suggesting that a drought
during this period would be more consequential than during other times.
\citet{cooperman2021natural} does briefly engage with the other two
measures of rainfall --- the three month measure from April to June and
the nine month measure from January to September --- in her
supplementary materials but only insofar as to observe the relationship
between rainfall during these times and whether a drought is declared.
The paper does not observe the implications of having below average
rainfall in these periods on drought declaration like was done in Table
1 of \citet{cooperman2021natural} using the six-month period data.

Additionally, because elections are held in October and voters may be
subject to recency bias, we became particularly interested in how
rainfall in the three months leading up to the election (July -
September) fits into this relationship. Because there was no specific
indicator for these three months provided in the dataset, we
extrapolated lower than average rainfall in July - September through
finding instances in which the nine month rainfall was below average but
the six month rainfall was not. Ideal data would allow for us to explore
this as a gross variable, but the data regarding rainfall in
\citet{cooperman2021natural} only shows rainfall relative to the average
for periods of time outside of the January-June period. As a result, we
were not able to find the exact amount of rain from July to September.
Despite this, our binary variable indicates whether rainfall was
significantly below average in the period leading up to the election.
While politicians cannot control the weather, hardship due to
below-average rainfall in the months immediately before the October
election could potentially impact outcomes.

\begin{table}[]
  \caption{Rainfall and Incumbent Outcomes}
  
% Table created by stargazer v.5.2.2 by Marek Hlavac, Harvard University. E-mail: hlavac at fas.harvard.edu
% Date and time: Sun, Dec 12, 2021 - 01:54:54
\begin{tabular}{@{\extracolsep{5pt}}lcccc} 
\\[-1.8ex]\hline 
\hline \\[-1.8ex] 
 & \multicolumn{4}{c}{\textit{Dependent variable:}} \\ 
\cline{2-5} 
\\[-1.8ex] & \multicolumn{4}{c}{Drought Emergency Declared} \\ 
\\[-1.8ex] & (1) & (2) & (3) & (4)\\ 
\hline \\[-1.8ex] 
 Below Average Rainfall (Jan-June) & 0.288$^{***}$ &  &  &  \\ 
  & (0.052) &  &  &  \\ 
  & & & & \\ 
 Below Average Rainfall (Apr-June) &  & 0.194$^{***}$ &  &  \\ 
  &  & (0.047) &  &  \\ 
  & & & & \\ 
 Below Average Rainfall (Jan-Sept) &  &  & 0.253$^{***}$ &  \\ 
  &  &  & (0.055) &  \\ 
  & & & & \\ 
 Below Average Rainfall (June-Sept) &  &  &  & $-$0.015 \\ 
  &  &  &  & (0.039) \\ 
  & & & & \\ 
 Mayor Election Year & 0.043 & $-$0.002 & 0.030 & 0.004 \\ 
  & (0.067) & (0.071) & (0.068) & (0.075) \\ 
  & & & & \\ 
 State/Fed Election Year & 0.006 & 0.011 & $-$0.004 & $-$0.032 \\ 
  & (0.080) & (0.076) & (0.080) & (0.079) \\ 
  & & & & \\ 
\hline \\[-1.8ex] 
Observations & 14,054 & 14,054 & 14,054 & 14,054 \\ 
R$^{2}$ & 0.267 & 0.243 & 0.257 & 0.214 \\ 
Adjusted R$^{2}$ & 0.208 & 0.182 & 0.197 & 0.151 \\ 
Residual Std. Error (df = 13013) & 0.445 & 0.452 & 0.448 & 0.461 \\ 
\hline 
\hline \\[-1.8ex] 
\textit{Note:}  & \multicolumn{4}{r}{$^{*}$p$<$0.1; $^{**}$p$<$0.05; $^{***}$p$<$0.01} \\ 
\end{tabular} 

\end{table}

To explore the relationship between rainfall and drought declarations,
as was done in Table 1 from the original paper, we repeat the analysis
for Model 1 but replace the six-month rainfall statistics with the
three-month (Model 2), nine-month (Model 3), and last three months
leading up to the election (Model 4). Our results are in line with the
findings from the six-month period: below average rainfall over the
periods from January-September and April-June are similarly positively
correlated with drought declarations. Interestingly, below-average
rainfall in the period immediately before the election (July-September)
is not significantly correlated with an increase in drought
declarations. This may be because this variable is binary and the sample
size is much smaller than for the other measures. It also could
potentially indicate that below-average rainfall in those three months
is not consequential for constituents or that because of the lag between
when a drought is declared and when aid is distributed, politicians do
not declare droughts during this time because they do not stand to
benefit politically from doing so. Thirdly, it could be the case that
politicians don't feel the need to declare a drought during this time
because it is no longer peak farming season, so the consequences of a
drought on farmers will be less severe. To draw any conclusions about
these potential explanations, more research and data are needed. Our
findings suggest that \citet{cooperman2021natural} chose an appropriate
time period to analyze droughts. Even without including the qualitative
data that explains the importance of the rainy period, it seems that
having below average rainfall between January and June better explains
the variation in the data than the other measures as it has the highest
R-squared value.

\section{ Rainfall Period and Reelection}
\label{sec:reelection}

\begin{table}[]
  \caption{Rainfall Amount and Drought Declaration}
  
% Table created by stargazer v.5.2.2 by Marek Hlavac, Harvard University. E-mail: hlavac at fas.harvard.edu
% Date and time: Sun, Dec 12, 2021 - 01:54:54
\begin{tabular}{@{\extracolsep{5pt}}lcccc} 
\\[-1.8ex]\hline 
\hline \\[-1.8ex] 
 & \multicolumn{4}{c}{\textit{Dependent variable:}} \\ 
\cline{2-5} 
\\[-1.8ex] & \multicolumn{2}{c}{Incumbent Runs} & \multicolumn{2}{c}{Incumbent Wins} \\ 
\\[-1.8ex] & (1) & (2) & (3) & (4)\\ 
\hline \\[-1.8ex] 
 Rainfall Compared to Average (effect of more rainfall) & 0.014 &  & $-$0.016 &  \\ 
  & (0.027) &  & (0.024) &  \\ 
  & & & & \\ 
 Below Average Rainfall &  & $-$0.064$^{***}$ &  & $-$0.037$^{*}$ \\ 
  &  & (0.021) &  & (0.022) \\ 
  & & & & \\ 
 Mayor's Vote Share in Previous Election & $-$0.029 & $-$0.029 & 0.336$^{***}$ & 0.334$^{***}$ \\ 
  & (0.087) & (0.086) & (0.094) & (0.094) \\ 
  & & & & \\ 
 Copartisan President & 0.061$^{*}$ & 0.060$^{*}$ & 0.038 & 0.040 \\ 
  & (0.036) & (0.036) & (0.033) & (0.034) \\ 
  & & & & \\ 
 Copartisan Governor & 0.004 & 0.004 & 0.007 & 0.006 \\ 
  & (0.025) & (0.025) & (0.024) & (0.024) \\ 
  & & & & \\ 
\hline \\[-1.8ex] 
Observations & 2,927 & 2,927 & 2,927 & 2,927 \\ 
R$^{2}$ & 0.016 & 0.018 & 0.030 & 0.030 \\ 
Adjusted R$^{2}$ & 0.010 & 0.011 & 0.023 & 0.024 \\ 
Residual Std. Error & 0.448 & 0.448 & 0.494 & 0.494 \\ 
\hline 
\hline \\[-1.8ex] 
\textit{Note:}  & \multicolumn{4}{r}{$^{*}$p$<$0.1; $^{**}$p$<$0.05; $^{***}$p$<$0.01} \\ 
\end{tabular} 

\end{table}

The above table and analysis suggest that the time of year in which
rainfall is observed as being ``below average'' does not significantly
affect whether a drought is declared. Because of this, we are interested
in directly observing the relationship between rainfall amount and both
whether an incumbent runs again, and if so, whether they win. The
analysis displayed in Table 3 of \citet{cooperman2021natural} focuses on
the binary variable ``below average rainfall'' and its effect on whether
an incumbent runs again and wins. While the dataset used includes a
continuous variable for rainfall relative to the average, the analysis
in the paper does not. We extend the analysis of
\citet{cooperman2021natural} by observing rainfall relative to other
years as a continuous variable to see whether the relationship seen
between an incumbent having ``below-average rainfall'' and running and
winning also holds when we observe rainfall as a continuous variable,
and specifically, whether politicians are rewarded for weather-related
prosperity that happens within their terms.

Interpreting Table 2, we do not find significant differences between
relative rainfall and whether or not an incumbent runs, nor between
rainfall and whether an incumbent wins. This could be for a number of
reasons including ``below average rainfall'' affecting communities more
than ``above average rainfall'' or perhaps that the categorical ``below
average rainfall'' variable matters but not rainfall variation more
generally.

\section{Conclusion}
\label{sec:conclusion}

Through our exploration of differential measures of rainfall, we
conclude that \citet{cooperman2021natural} both made the optimal and
logical choice in choosing the six-month rainy period as the main period
of rainfall observation. The results do not differ significantly from
the three and nine month periods. Additionally, we find that measuring
rainfall using a continuous variable rather than a binary ``below
average'' variable eliminates the significant effects of rainfall on
incumbents running for and winning reelection. This contradicts the
findings of \citet{cooperman2021natural} but does not necessarily
undermine the results and could potentially be dismissed because of
droughts being more relevant than periods of increased rainfall.

Despite our confusion about the nebulous nature of the ``lame duck''
variable and the possibility that this interfered with both our analysis
and the analysis of \citet{cooperman2021natural}, we believe that
regardless of whether the variable indicates a true lame duck or simply
a term-limited politician, that there are interesting conclusions to be
drawn about the behaviors of politicians who do not experience a
reelection incentive.

Replication papers like this one are crucial for ensuring the validity
and the significance of academic findings. Sometimes, even if results
are replicable, additional robustness checks, changes in modeling, or
unique approaches can change the significance of results, or even the
results themselves. With political science topics having great
significance for policy making around the world, it is essential that we
get these questions right or at least right as we possibly could.
Replication and peer review strengthen the academic literature and the
legitimacy of political science as a field. Even when, like in this
paper, there are minimal faults found with the original piece,
replication serves as an important step for ensuring validity of
results.

\newpage

\bibliographystyle{agsm}
\bibliography{references.bib}

\end{document}
